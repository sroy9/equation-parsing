  Understanding mathematical relations in text involves identifying
  key entities, and detecting the mathematical equation relating these
  entities. Sentences often uses mathematical jargon, and describes
  composition of mathematical expressions, which require non-trivial
  reasoning. For example, in the sentence {\em``The difference of twice a
  number and 3 is five more than thrice another number.''}, we need to
  know the meaning of ``difference'', as well understand the
  composition that ``five'' needs to be added after ``three'' is
  multiplied with the number.

  Grounding of variables to entities in text is necessary for deep
  reasoning, as well as analysis across multiple sentences. However,
  the text often does not mention explicitly what the variables in the
  resulting equation represent. For example, the sentence {\em``Flying
  with the wind , a bird was able to make 150 kilometers per hour.''}
  describes a mathematical relation between the speed of bird and the
  speed of wind, without mentioning ``speed'' explicitly. There is
  also ambiguity regarding mention choice while grounding
  variables. For example, in the following sentence-equation pair

  \setlength{\tabcolsep}{6pt}
  \begin{table}[H]
   \centering \small
   \begin{tabular}{|lp{5cm}|}
     \hline Example 1 &  \\
     \hline Sentence & {\em City Rentals rent an intermediate-size car
  for 18.95 dollars plus 0.21 per mile.} \\
     \hline Equation & $V_1=18.95+0.21*V_2$ \\
     \hline 
   \end{tabular}
   \label{tab:example1}
  \end{table}
  
  \noindent the variable $V_1$ could be grounded to ``City Rentals'',
  ``rent'', or even ``City Rentals rent an intermediate-size car''.

  In this paper, we introduce a task where given a sentence describing
  a mathematical relation between entities, the goal is to predict an
  equation representing the relation, and to determine what the
  variables refer to. As seen in example 2, there are often multiple
  spans of text which can be valid grounding of a variable. Here,
  $V_1$ can be correctly grounded to both ``the first one'' or ``both
  numbers''. Since exact segmentation of mentions can be ambiguous, we
  ground variables to tokens in the sentence, which are either nouns,
  verbs or adjectives. A grounding is said to be valid if the system
  grounds to a token within a gold grounded span.
  
  \setlength{\tabcolsep}{6pt} 
  \begin{table}[H] 
         \centering 
         \small 
         \begin{tabular}{|lp{5cm}|} \hline
         Example 2 & \\ \hline 
         Sentence & {\em If 10 is added to two numbers,
         the first one will be 5 more than thrice the second.} \\ \hline
         Equation & $V_1+10=5+3*(V_2+10)$\\ \hline 
         Grounding & $V_1= \{\text{``the first one'', ``two numbers''}\}$
         $V_2= \{\text{``the second'', ``two numbers''}\}$\\ 
        \hline 
        \end{tabular} 
        \label{tab:example2} 
  \end{table}
  
  We propose a novel method for equation prediction and variable
  grounding. First, we detect and normalize all quantities in the
  sentence, and prune the irrelevant quantities, which do not take
  part in the relation. Next we develop a structured prediction model
  to jointly predict variable grounding and equation generation. Since
  there can be multiple valid groundings for a problem, we model the
  best grounding to be latent, and let the model figure out which
  grounding is best for it, to generate equations.
  
  Our inference algorithm relies heavily on the idea of an equation
  parse of a sentence. The parse can effectively capture
  compositionality, can directly generate the necessary equation from
  text, and provides a way to explore the space of all equations
  efficiently.

  We develop and annotate datasets for evaluation and show that our
  method can handle the equation generation and ground task quite
  well.

  The next section presents some related work on mathematical
  reasoning and automatic problem solvers. We then present our
  equation parse representation, and
  

  Understanding text used in algebra word problems involves
  identifying key entities, and detecting the mathematical relations
  among these entities. The problem statement often uses mathematical
  jargon, and describes composition of mathematical expressions, which
  require non-trivial reasoning. Consider the following sentence :
  \noindent
  \fbox{
    \parbox{0.45\textwidth} {\small
      \noindent
      \textbf{Example $1$}
      \newline
      {\em The difference of twice a number and 3 is five more than
        thrice another number. }
      \newline
    } } 
  Here, we need to identify that ``difference'' is a math expression,
  denoting a subtraction between its arguments, which are ``twice a
  number'' and ``3''. Composition understanding is needed to know that
  ``five'' is to be added after $3$ is multiplied with ``another
  number''.

  Grounding of variables to entities in text is necessary for deep
  reasoning, as well as analysis across multiple sentences. However,
  the text often does not mention explicitly what the variables in the
  resulting equation represent. Consider the sentence-equation pair
  below :
  \noindent
  \fbox{
    \parbox{0.45\textwidth} {\small
      \noindent
      \textbf{Example $2$}
      \newline
      $\mathrm{Q}$:{\em Mrs. Lloyd plans on spending 370 dollars on 10
        meals and 3 nights in a hotel.}
      \newline
      $\mathrm{A}$:{$10V_1+3V_2=370$ } } } 
  The variables $V_1$ and $V_2$ represent the cost of a meal and the
  cost of spending a night, respectively, none of which is explicitly
  mentioned. We need to determine from ``spending'' that the relation
  is related to currency.

  Again, there are cases where irrelevant numbers are mentioned in the
  text, as in
  \noindent
  \fbox{
    \parbox{0.45\textwidth} {\small
      \noindent
      \textbf{Example $3$}
      \newline
      $\mathrm{Q}$:{\em Bob invested 22,000 dollars , part at 18 \%
        and part at 14 \%.}
      \newline
      $\mathrm{A}$:{$V_1+V_2=22,000$ } } } 
  Though the phrases ``part at 18\%'' and ``part at 14 \%'' assist in
  grounding variables to the text, they mention quantities which are
  not relevant to the relation mentioned, and we need to learn to
  disregard them while generating equations.

  In this paper, we develop a methodology to ... (dependent on
  implementation)

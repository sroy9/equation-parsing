\begin{figure*}
\begin{center}
  \begin{tikzpicture}[framed]
  \node[draw=none] (1) {If};
  \node[draw, right=0.1cm of 1] (2) {10};
  \node[draw=none, right=0.1cm of 2] (3) {is added to two numbers,};
  \node[draw, right=0.1cm of 3] (4) {the first one};
  \node[draw=none, right=0.1cm of 4] (5) {will be};
  \node[draw, right=0.1cm of 5] (6) {5};
  \node[draw=none, right=0.1cm of 6] (7) {more than};
  \node[draw, right=0.1cm of 7] (8) {thrice};
  \node[draw, right=0.1cm of 8] (9) {the second};
  \node[circle,draw,fill=white,text=black] (14) at (7, 1) {$V_1$};
  \node[circle,draw,fill=white,text=black] (13) at (13, 1) {$V_2$};
  \node[circle,draw,fill=white,text=black] (10) at (11, 2) {*};
  \node[circle,draw,fill=white,text=black] (11) at (9,2.5){+};
  \node[circle,draw,fill=white,text=black] (12) at (7.5,3){=};
  \draw[->,solid,-triangle 90,fill=black] (10) to (8);
  \draw[->,solid,-triangle 90,fill=black] (11) to (6);
  \draw[->,solid,-triangle 90,fill=black] (11) to (10);
  \draw[->,solid,-triangle 90,fill=black] (12) to (11);
  \draw[->,solid,-triangle 90,fill=black] (12) to (14);
  \draw[->,solid,-triangle 90,fill=black] (14) to (4);
  \draw[->,solid,-triangle 90,fill=black] (13) to (9);
  \draw[->,solid,-triangle 90,fill=black] (10) to (13);
  \draw[bend right,->]  (2) to node [auto] {Modifier} (4);
  \draw[bend right,->]  (2) to node [auto] {Modifier} (9);
  \end{tikzpicture}
\end{center}
\caption{Example of an Equation parse}
\label{fig:eqparse}
\end{figure*}

  In this section, we introduce an equation parse for a
  sentence. First we introduce a few terms.

  \noindent \textbf{Trigger} Given a sentence S mentioning a
  mathematical relation, a trigger represents either a quantity
  expressed in S, or a span of text in S, which represents a variable
  for the output equation. A quantity trigger is a tuple $(q, s)$,
  where $q$ is the numeric value of the quantity mentioned in text,
  and $s$ is the span of text from $S$ which mentions the quantity. A
  variable trigger is also a tuple $(l, s)$, where $l$ represents a
  naming of the variable, and $s$, as before, represents the span of
  text representing the variable. For example, for the sentence in Fig
  \ref{fig:eqparse}, the spans ``10'', ``5'', and ``thrice'' generate
  quantity triggers, whereas ``the first one'' and ``the second''
  generate variable triggers. The same span of text can generate
  multiple triggers.  For example, the span ``two numbers'' represent
  a quantity, but can be a valid grounding for both the variables in
  text. Therefore, it generates $3$ triggers - (2.0, ``two numbers''), 
  ($V_1$, ``two numbers''), and ($V_2$, ``two numbers'').

  \noindent \textbf{Equation Tree} An equation tree is a binary tree
  whose leaves represent triggers, and internal nodes (except the
  root) are labeled with one of the following operations -- addition,
  subtraction, multiplication, division. In addition, for nodes which
  are labeled with subtraction or division, we maintain a separate
  variable to determine order of its children. The root of the tree is
  always labeled with the operation equal. In Fig \ref{fig:eqparse},
  the equation tree is constructed on the triggers generated by ``the
  first one'', ``5'', ``thrice'', and ``the second''. 

  An equation tree also gives a natural representation for an
  equation. Each node $n$ in an equation tree represents an expression
  $Expr(n)$, and the label of the parent node determines how the
  expressions of its children need to be composed to construct its own
  expression. Let us denote the operation label for a non-leaf node
  $n$ to be $Op(n)$, which can take values from $\{+, -, *, /, =\}$
  and the order of a node $n$'s children by $Order(n)$, which takes
  values ${lr, rl}$. For trigger node $n$, the expression $Expr(n)$
  represents the variable label, if $n$ is a variable trigger, and the
  numeric value of the quantity, if its a quantity trigger. Finally,
  we use $lc(n)$ and $rc(n)$ to represent the left and the right child
  of node $n$ respectively. The relation to generate the expression
  can then be wriiten recursively as follows :
  
  For all non-leaf node $n$ of an equation tree, we have
  \begin{multline}
    Expr(n) = \\
    Expr(lc(n)) Op(n) Expr(rc(n)) \\
    \mbox{if } Op(n) \in \{+,*, =\} \\
    Expr(lc(n)) Op(n) Expr(rc(n)) \\
    \mbox{if } Op(n) \in \{-,/\}, Order(n)=lr \\
    Expr(rc(n)) Op(n) Expr(lc(n)) \\
    \mbox{if } Op(n) \in \{-,/\}, Order(n)=rl \\
  \end{multline}

  Referring to the equation tree in \ref{fig:eqparse}, the node marked
  ``*'' represents $3*V_2$, the node marked ``+'' represents
  $5+3*V_2$, and finally the root represents $V_1=5+3*V_2$.
  
  The leaves in an equation tree form an ordered list of triggers,
  that is, the order of the leaves used in an equation tree is the
  same as in which those triggers were found in text. Hence equation
  trees implicitly provide a constituent parse like structure for the
  sentence. The assumption behind equation trees is that if we need to
  perform an operation between two expressions, then those expression
  will be mentioned one after the other, that is, there will be no
  other expression from the equation, which is mentioned between
  them. For example, in Fig \ref{fig:eqparse}, ``thrice'' and ``the
  second'' are found next to each other, ``5'' and ``thrice the
  second'' does not have any trigger mentioned in the span between
  them. 

  Although most cases of mathematical expressions satisfy the
  aforementioned assumption, there are certain exceptions to this
  rule.  Certain numbers in the text modify both variables in the same
  manner, and they might not be mentioned in the viscinity of grounded
  variable spans. In our running example, the number ``10'' needs to
  be added to both variables, the grounding shown in Fig
  \ref{fig:eqparse} keeps ``10'' and $V_2$ far from each other. In
  order to support such variable modification, we introduce the
  concept of modifiers, which we define next.

  \noindent \textbf{Modifiers} Modifiers are quantities or expressions
  mentioned in text, which modify both variables of an equation in a
  similar way. For example, in ``When you double two numbers, their
  sum is 30'', the term ``double'' determines that $2$ needs to be
  multiplied with both variables. Again, in ``In 5 years, John will be
  twice as old as his brother.'', ``10'' needs to be added to the ages
  of both John and his brother, before we can use the relation
  expressed in the later part of the text. We also associate a label
  with each modifier to know how it is changing the variables in the
  sentence. For example, ``double'' in the first example was a
  multiplicative modifier, whereas ``10'' in the second example is an
  additive modifier. In our problem, we restrict modifiers to comprise
  only single quantity triggers. There are six possible ways it can
  modify the variables in the problem, - addition, subtraction (both
  orders), multiplication and division (again, both orders).

  \noindent \textbf{Equation Parse} An equation parse of a sentence S
  is a 4-tuple $(T, E, M)$, where $T$ represents a subset of triggers
  extracted from S, $E$ represents an equation tree formed with the
  set $T$ as leaves, and $M$ represents a modifier. Given an equation
  parse of a sentence, the equation represented by it is the
  expression generated by the root of $T$, followed by a modification
  of the variables by the modifier $M$.



